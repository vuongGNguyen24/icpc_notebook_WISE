% Written by Anders Sjoqvist and Ulf Lundstrom, 2009
% The main sources are: tinyKACTL, Beta and Wikipedia

\chapter{Mathematics}

\section{Equations}
In general, given an equation $Ax = b$, the solution to a variable $x_i$ is given by
\[x_i = \frac{\det A_i'}{\det A} \]
where $A_i'$ is $A$ with the $i$'th column replaced by $b$.

\section{Sums}

\begin{align*}
    c^a + c^{a+1} + \dots + c^{b} &= \frac{c^{b+1} - c^a}{c-1}, c \neq 1 \\
    c^1 + 2c^2 + \dots + nc^{n} &= \frac{nc^{n+2} - (n + 1)c^{n + 1} + c}{(c-1)^2}, c \neq 1 \\
	1 + 2 + 3 + \dots + n &= \frac{n(n+1)}{2} \\
	1^2 + 2^2 + 3^2 + \dots + n^2 &= \frac{n(2n+1)(n+1)}{6} \\
	1^3 + 2^3 + 3^3 + \dots + n^3 &= \frac{n^2(n+1)^2}{4} \\
\end{align*}

% \section{Series}
% $$e^x = 1+x+\frac{x^2}{2!}+\frac{x^3}{3!}+\dots,\,(-\infty<x<\infty)$$
% $$\ln(1+x) = x-\frac{x^2}{2}+\frac{x^3}{3}-\frac{x^4}{4}+\dots,\,(-1<x\leq1)$$
% $$\sqrt{1+x} = 1+\frac{x}{2}-\frac{x^2}{8}+\frac{2x^3}{32}-\frac{5x^4}{128}+\dots,\,(-1\leq x\leq1)$$
% $$\sin x = x-\frac{x^3}{3!}+\frac{x^5}{5!}-\frac{x^7}{7!}+\dots,\,(-\infty<x<\infty)$$
% $$\cos x = 1-\frac{x^2}{2!}+\frac{x^4}{4!}-\frac{x^6}{6!}+\dots,\,(-\infty<x<\infty)$$

\section{Probability theory}
Let $X$ be a discrete random variable with probability $p_X(x)$ of assuming the value $x$. It will then have an expected value (mean) $\mu=\mathbb{E}(X)=\sum_xxp_X(x)$ and variance $\sigma^2=V(X)=\mathbb{E}(X^2)-(\mathbb{E}(X))^2=\sum_x(x-\mathbb{E}(X))^2p_X(x)$ where $\sigma$ is the standard deviation. If $X$ is instead continuous it will have a probability density function $f_X(x)$ and the sums above will instead be integrals with $p_X(x)$ replaced by $f_X(x)$.

\subsection{Discrete distributions}

\subsubsection{First success distribution}
The number of trials needed to get the first success in independent yes/no experiments, each which yields success with probability $p$ is $\textrm{Fs}(p),\,0\leq p\leq1$.
\[p(k)=p(1-p)^{k-1},\,k=1,2,\dots\]
\[\mu = \frac1p,\,\sigma^2=\frac{1-p}{p^2}\]

\subsubsection{Poisson distribution}
The number of events occurring in a fixed period of time $t$ if these events occur with a known average rate $\kappa$ and independently of the time since the last event is $\textrm{Po}(\lambda),\,\lambda=t\kappa$.
\[p(k)=e^{-\lambda}\frac{\lambda^k}{k!}, k=0,1,2,\dots\]
\[\mu=\lambda,\,\sigma^2=\lambda\]

\subsection{Continuous distributions}

\subsubsection{Uniform distribution}
If the probability density function is constant between $a$ and $b$ and 0 elsewhere it is $\textrm{U}(a,b),\,a<b$.
\[f(x) = \left\{
\begin{array}{cl}
\frac{1}{b-a} & a<x<b\\
0 & \textrm{otherwise}
\end{array}\right.\]
\[\mu=\frac{a+b}{2},\,\sigma^2=\frac{(b-a)^2}{12}\]


\section{Linear algebra}
    \kactlimport{Gauss.h}
    \kactlimport{GaussBinary.h}
    \kactlimport{MatrixInverseBinary.h}